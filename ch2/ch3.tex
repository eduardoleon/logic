\chapter{Teorema de solidez}

\begin{notation}
Denotamos por $\mathscr F$ el conjunto de frases bien formadas de $L$.
\end{notation}

\begin{notation}
Denotamos el conjunto de valores de verdad $\Omega = \{ V, F \}$.
\end{notation}

\begin{definition}
Una \textit{valoración} en $L$ es una función $v : \mathscr F \to \Omega$ tal que
\begin{itemize}
    \item $v(A) \ne v(\neg A)$
    \item $v(A \to B) = F$ si y sólo si $v(A) = V$ y $v(B) = F$.
\end{itemize}
\end{definition}

\begin{remark}
Una valoración es una asignación de valores de verdad para las proposiciones atómicas.
\end{remark}

\begin{definition}
Una \textit{tautología} es una frase bien formada $A$ tal que $v(A) = V$ para toda valoración.
\end{definition}

\begin{proposition}
(Teorema de solidez) Todo teorema de $L$ es una tautología.
\end{proposition}

\begin{prove}
Por inducción estructural en la demostración de $\vdash_L A$:

\begin{itemize}
    \item Si $A$ es un axioma de $L$, entonces es una tautología (ejercicio).
    
    \item Si $A$ se deduce por modus ponens, existe una frase bien formada $B$ tal que
    \begin{itemize}
        \item $\vdash_L B$
        \item $\vdash_L B \to A$
    \end{itemize}
    
    Inductivamente, $B$ y $B \to A$ son tautologías. Sea $v : \mathscr F \to \Omega$ una valoración cualquiera. Entonces se tiene $v(B) = v(B \to A) = V$, lo cual implica que $v(A) = V$.
\end{itemize}
\end{prove}

\begin{exercise}
Pruebe que todo axioma de $L$ es una tautología.
\end{exercise}

\begin{solution}
Construyamos las tablas de verdad de L1, L2 y L3.

\begin{center}
\begin{tabular}{cc|c|c|c|c}
    $A$ & $B$ & $A \to B$ & $\neg B \to \neg A$ & L1 & L3 \\
    \hline
    V & V & V & V & V & V \\
    V & F & F & F & V & V \\
    F & V & V & V & V & V \\
    F & F & V & V & V & V \\
\end{tabular}
\end{center}

\begin{center}
\begin{tabular}{ccc|c|c|c}
    $A$ & $B$ & $C$ & $A \to (B \to C)$ & $\Lbh ABC$ & L2 \\
    \hline
    V & V & V & V & V & V \\
    V & V & F & F & F & V \\
    V & F & V & V & V & V \\
    V & F & F & V & V & V \\
    F & V & V & V & V & V \\
    F & V & F & V & V & V \\
    F & F & V & V & V & V \\
    F & F & F & V & V & V \\
\end{tabular}
\end{center}

Sea $P$ un axioma de $L$ y sea $v : \mathscr F \to \Omega$ una valoración.
\begin{itemize}
    \item Si $P = \La AB$, los valores de $v(A)$ y $v(B)$ determinan una única fila en la primera tabla de verdad. La entrada en la columna L1 es el valor de $v(P)$.
    
    \item Si $P = \Lb ABC$, los valores de $v(A)$, $v(B)$ y $v(C)$ determinan una única fila en la segunda tabla de verdad. La entrada en la columna L2 es el valor de $v(P)$.
    
    \item Si $P = \Lc AB$, los valores de $v(A)$ y $v(B)$ determinan una única fila en la primera tabla de verdad. La entrada en la columna L3 es el valor de $v(P)$.
\end{itemize}

En todos los casos, $v(P) = V$. Por lo tanto, $P$ es una tautología.
\end{solution}
